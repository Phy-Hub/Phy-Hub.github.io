\frontmatter
%███████████████████████████████████████████████████████████████████
%███████████████████████████████████████████████████████████████████
%███████████████████████████████████████████████████████████████████
\chapter{Intro}

Classical physics accurately predicts the movement of objects in our everyday world. However, as objects approach the speed of light, there are effects that it does not account for, which were once negligible but become increasingly significant with increasing speed. Two observers moving at high speeds relative to each other become unable to agree on the lengths of objects and times in which events occur. Crucially, one thing remains constant: the speed of light relative to each observer is the same, regardless of their movement. We must redefine our understanding of time and space to find out how this can be. This is where special relativity comes into play.

Special relativity is a vital part of particle, nuclear, and astrophysics. It is used to calculate the power output of nuclear reactions and measure galaxy speeds via effects on the emitted light's frequency. It is also needed for precision timekeeping on fast-moving GPS satellites to calculate positions accurately.

Despite being well-established and essential in modern physics, special relativity continues to confuse even those well-versed in physics and mathematics. It can be as confusing to visualize and understand as quantum mechanics is. The theory requires letting go of some deeply held intuitions about the absolute nature of space and time that differ from our usual everyday experience of the physical world.

This handbook is meant to offer an intuitive, visual, and comprehensive overview. The aim is to maintain simplicity and avoid unnecessary abstraction in visual and conceptual explanations. For example, time in diagrams is shown either through the use of animations or by multiple diagrams given one after another in a timeline. This approach avoids the use of unnecessary space-time diagrams, which represent time on one of the axes and make special relativity more abstract and confusing to visualize.

The first chapter is readable even for those without a math or physics background. It provides a non-mathematical, visual overview of the concepts. This chapter builds an intuitive understanding of the core ideas step-by-step, providing a solid foundation for developing the essential mathematical framework in later sections. The mathematical portion will begin with deriving how space and time coordinates function in special relativity. This will be followed by velocity addition, the Doppler effect, and the energy-momentum relationship. In addition to these topics, we will also explore some interesting subjects not typically covered in standard resources. These include how a source’s emitted light is concentrated in the direction of its movement, called aberration, and how a delay in light signals leads to retarded source positions and fields. These topics will probably not mean much to you yet, but they will give interesting insights.
\vspace{1cm} \newline
Now, let us begin!

%the Liénard-Wiechert-Potentials, maybe quaternions in SR if you are lucky, and so on.
% Your momma did not raise you right, so it is my turn to whip you into shape. By the end of this book, you will also be able to understand yourself a little more (because there will be a subsection on retarded jerks). You already have something in common with this topic of relativity because you are special!... The only thing that can undo Lorentz length contraction is your momma
% This book is not an inertial frame of reference unless in free fall (but gravity, which is for a different day). Damage caused to this book by allowing it to enter a short-lived inertial frame of reference before making an abrupt exit is not covered by our returns policy (do not throw the book at the ground!). \\

\newpage