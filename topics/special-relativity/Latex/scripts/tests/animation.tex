\documentclass{article}
\usepackage{tikz}
\usepackage{animate}
\usepackage{fp} % Add this line
\usepackage{graphicx} % Add this line

\begin{document}
\begin{figure}
\scalebox{0.5}{ % Scale the figure to 50% of its original size
\begin{animateinline}[autoplay,loop]{1} % 1 frames per second
  \multiframe{11}{iShift=0+1}{ % 101 frames, shift from 0 to 10
    \FPeval\result{clip(\iShift)} % Calculate the shift
    \begin{tikzpicture}
      \def\numArrows{40} % number of arrows
      \def\circleRadius{3} % radius of the circle
      \def\squareSide{2.1*\circleRadius} % side of the square
      \def\smallCircleRadius{0.2*\circleRadius} % radius of the small circles

    \path (-\squareSide,0) -- (10+\squareSide,0);

    % Draw the square and fill it with black
    \fill[black] (-\squareSide/2 + \result,-\squareSide/2) -- (\squareSide/2 + \result,-\squareSide/2) -- (\squareSide/2 + \result,\squareSide/2) -- (-\squareSide/2 + \result,\squareSide/2) -- cycle;

    % Draw the small circles
    \fill[black] (-\squareSide/3 + \result,-\squareSide/2) circle (\smallCircleRadius);
    \fill[black] ( \squareSide/3 + \result,-\squareSide/2) circle (\smallCircleRadius);

    % Draw the circle and fill it with white
    \fill[white] (0 + \result,0) circle (\circleRadius);

    % Draw the arrows
    \foreach \i in {1,2,...,\numArrows} {
        \draw[-latex] (\i*360/\numArrows:0*\circleRadius) -- (\i*360/\numArrows:0.4*\circleRadius);
    }

      \fill[red] (0,0) circle (0.1);

    \end{tikzpicture}
  }
\end{animateinline}
} % End scalebox
\caption{zdfg}
\end{figure}

\end{document}
